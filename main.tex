\documentclass[12pt, a4paper]{article}
\usepackage[utf8]{inputenc}
\usepackage[T1]{fontenc}
\usepackage[english,brazil]{babel}
\usepackage{indentfirst}
\usepackage{graphicx}
\usepackage{amsmath}
\usepackage[left=3cm, top=3cm, right=2cm, bottom=2cm]{geometry}
\usepackage{setspace}
%links%
\usepackage{hyperref}
\setlength{\parindent}{1.25cm}

\begin{document}
% capa
\begin{titlepage}
\begin{center}
    \large
     {\bf Centro Estadual de Educação Tecnológica Paula Souza} \\
     {\bf Faculdade de Tecnologia Baixada Santista
Rubens Lara} \\ 
    {\bf Curso Superior de Tecnologia em Ciência de Dados} \\
    
    \vspace{215pt}
        {\bf Matemática Básica} \\
    \vspace{10pt}
        {\Large \bf Determinante de matriz 4x4}\\
        
    \vspace{100pt}

    \vfill
        {\large  \bf Autor} \\
        {\large  \bf Fernando Gomes Cruz} 
    \vfill
        \textbf{{\large Santos}\\
        {\large 2023}}
        
\end{center}
\end{titlepage}

\tableofcontents

\newpage

\section{Introdução}

Esse trabalho consiste em demonstrar o cálculo da determinante de uma matriz 4x4 utilizando a fórmula de Leibniz. Esse tema é um importante conceito da álgebra linear, que possui diversas aplicações práticas em áreas como engenharia, física e computação. O conhecimento da fórmula para o cálculo da determinante de uma matriz é crucial para o cientista de dados, pois permite a compreensão das propriedades fundamentais de sistemas lineares, tais como a solução de sistemas de equações lineares, o cálculo de autovalores e autovetores, e a análise da geometria das transformações lineares, aspectos essenciais em diversas aplicações práticas em ciência de dados. 
\section{Dedução da determinante 4x4 a partir da fórmula de Leibniz}

Nesta seção, é apresentado o passo a passo do cálculo da determinante de uma matriz 4x4 utilizando a fórmula de Leibniz. Mostramos como é possível aplicar os conceitos da fórmula para calcular os produtos de todas as permutações possíveis dos elementos da matriz. Com essa análise, poderemos chegar a uma fórmula geral que permite o cálculo da determinante de qualquer matriz 4x4.
\subsection{Fórmula de Leibniz}
A fórmula de Leibniz para determinantes para uma matriz quadrada é dada por:

$$
\det(A) = \sum_{\sigma \in S_n} \mathrm{sgn}(\sigma) \prod_{i=1}^n a_{i,\sigma(i)}
$$

\noindent onde,
\begin{itemize}
\item $A$ é uma matriz quadrada $n$ por $n$ com suas entradas $a_{i,j}$
\item $S_n$ é o conjunto de todas as permutações dos números $1, 2, ..., n,$ e $sgn(\sigma)$ é o sinal da permutação sigma
\item A somatória é tomada sobre todas as permutações em $S_n$,  e pra cada permutação sigma, tomamos o produto das entradas $a_{i,\sigma(i)}$ de acordo com a diagonal de $A$ com o sinal  $sgn(\sigma)$ 
\end{itemize}

\subsection{Permutações}
Permutações são arranjos de elementos que seguem uma ordem específica. Em outras palavras, são reorganizações dos elementos de um conjunto. Na fórmula de Leibniz para o cálculo da determinante de uma matriz, as permutações são utilizadas para especificar a ordem em que as entradas da matriz devem ser multiplicadas e somadas. Cada permutação corresponde a uma maneira diferente de organizar as linhas e colunas da matriz, e a sua contribuição para o valor da determinante depende de sua paridade.
\\
\par A primeira etapa é definir todas as permutações dos números $1, 2, 3, 4$ já que se trata de uma matriz $4 \times 4$.
\par O número de permutações é dado por $n!$, $4! = 24$, logo serão 24 permutações.
\begin{equation*}
\begin{aligned}
S_4 = & \{'1234', '1243', '1324', '1342', '1432', '1423', '2134', '2143', \\
    & '2314', '2341', '2431', '2413', '3214', '3241', '3124', '3142', \\
    & '3412', '3421', '4231', '4213', '4321', '4312', '4132', '4123'\}
\end{aligned}
\end{equation*}
\subsection{Inversões}

O elemento $sgn(\sigma)$ pode ser dado por $sgn(\sigma) = (-1)^{N(\sigma)}$, onde $N(\sigma)$ é o número de inversões em $\sigma$.

\par Dado a permutação $\sigma$ de tamanho $n$, o par $(i, j)$, onde $1 \leq i < j \leq n$ é chamado de \textbf{inversão}, se $i<j$ e $\sigma(i) > \sigma(j)$. A inversão é indicado por um par ordenado contendo os locais $(i, j)$ ou os elementos $(\sigma(i), \sigma(j))$
\par Observação: Se $\sigma$ não contem inversões, então $\sigma = \epsilon$.
\\ \\
Como exemplo, tendo Inv como o conjunto de Inversões: \\\\
$\sigma =\left(\begin{array}{cccc} 1 & 2 & 3 & 4\\ 4 & 3 & 2 & 1 \end{array}\right)$, Inv = $\{(1, 2), (1, 3), (1, 4), (2, 3), (2, 4), (3, 4)\}$
\\\\
$N(\sigma) = |Inv| = 6$
\\\\
$sgn(\sigma) = (-1)^{N(\sigma)} = (-1)^6 = 1$
\subsubsection{Inversões e sinal para todas as permutações}

\par Temos que calcular o sinal para todas as permutações de $S_4$.\\
\\
\noindent
$\sigma = \left(\begin{array}{cccc} 1 & 2 & 3 & 4\\ 1 & 2 & 3 & 4 \end{array}\right)$,\\  $Inv = \{\}, sgn(\sigma) = (-1)^{N(\sigma)} = (-1)^0 = 1$\\
\\
\noindent
$\sigma =\left(\begin{array}{cccc} 1 & 2 & 3 & 4\\ 1 & 2 & 4 & 3 \end{array}\right)$,\\ $Inv = \{(3, 4)\}, sgn(\sigma) = (-1)^{N(\sigma)} = (-1)^1 = -1$\\
\\
\noindent
$\sigma =\left(\begin{array}{cccc} 1 & 2 & 3 & 4\\ 1 & 3 & 2 & 4 \end{array}\right)$,\\ $Inv = \{(2,3)\}, sgn(\sigma) = (-1)^{N(\sigma)} = (-1)^1 = -1$\\
\\
\noindent
$\sigma =\left(\begin{array}{cccc} 1 & 2 & 3 & 4\\ 1 & 3 & 4 & 2 \end{array}\right)$,\\ $Inv = \{(2, 4), (3,4)\}, sgn(\sigma) = (-1)^{N(\sigma)} = (-1)^2 = 1$\\
\\
\noindent
$\sigma =\left(\begin{array}{cccc} 1 & 2 & 3 & 4\\ 1 & 4 & 3 & 2 \end{array}\right)$,\\ $Inv = \{(2,3), (2,4), (3, 4)\}, sgn(\sigma) = (-1)^{N(\sigma)} = (-1)^3 = -1$\\
\\
\noindent
$\sigma =\left(\begin{array}{cccc} 1 & 2 & 3 & 4\\ 1 & 4 & 2 & 3 \end{array}\right)$,\\  $Inv = \{(2,3), (2, 4)\}, sgn(\sigma) = (-1)^{N(\sigma)} = (-1)^2 = 1$\\
\\
\noindent
$\sigma =\left(\begin{array}{cccc} 1 & 2 & 3 & 4\\ 2 & 1 & 3 & 4 \end{array}\right)$,\\  $Inv = \{(1,2)\}, sgn(\sigma) = (-1)^{N(\sigma)} = (-1)^1 = -1$\\
\\
\noindent
$\sigma =\left(\begin{array}{cccc} 1 & 2 & 3 & 4\\ 2 & 1 & 4 & 3 \end{array}\right)$,\\  $Inv = \{(1,2), (3,4)\}, sgn(\sigma) = (-1)^{N(\sigma)} = (-1)^2 = 1$\\
\\
\noindent
$\sigma =\left(\begin{array}{cccc} 1 & 2 & 3 & 4\\ 2 & 3 & 1 & 4 \end{array}\right)$,\\  $Inv = \{(1, 3), (2, 3)\}, sgn(\sigma) = (-1)^{N(\sigma)} = (-1)^2 = 1$\\
\\
\noindent
$\sigma =\left(\begin{array}{cccc} 1 & 2 & 3 & 4\\ 2 & 3 & 4 & 1 \end{array}\right)$,\\  $Inv = \{(1, 4), (2, 4), (3, 4)\}, sgn(\sigma) = (-1)^{N(\sigma)} = (-1)^3 = -1$\\
\\
\noindent
$\sigma =\left(\begin{array}{cccc} 1 & 2 & 3 & 4\\ 2 & 4 & 3 & 1 \end{array}\right)$,\\  $Inv = \{(1, 4), (2, 3), (2, 4), (3, 4)\}, sgn(\sigma) = (-1)^{N(\sigma)} = (-1)^4 = 1$\\
\\
\noindent
$\sigma =\left(\begin{array}{cccc} 1 & 2 & 3 & 4\\ 2 & 4 & 1 & 3 \end{array}\right)$,\\  $Inv = \{(1, 3), (2, 3), (2, 4)\}, sgn(\sigma) = (-1)^{N(\sigma)} = (-1)^3 = -1$\\
\\
\noindent
$\sigma =\left(\begin{array}{cccc} 1 & 2 & 3 & 4\\ 3 & 2 & 1 & 4 \end{array}\right)$,\\  $Inv = \{(1, 2), (1, 3), (2, 3)\}, sgn(\sigma) = (-1)^{N(\sigma)} = (-1)^3 = -1$\\
\\
\noindent
$\sigma =\left(\begin{array}{cccc} 1 & 2 & 3 & 4\\ 3 & 2 & 4 & 1 \end{array}\right)$,\\  $Inv = \{(1, 2), (1, 4), (2, 4), (3, 4)\}, sgn(\sigma) = (-1)^{N(\sigma)} = (-1)^4 = 1$\\
\\
\noindent
$\sigma =\left(\begin{array}{cccc} 1 & 2 & 3 & 4\\ 3 & 1 & 2 & 4 \end{array}\right)$,\\  $Inv = \{(1, 2), (1, 3)\}, sgn(\sigma) = (-1)^{N(\sigma)} = (-1)^2 = 1$\\
\\
\noindent
$\sigma =\left(\begin{array}{cccc} 1 & 2 & 3 & 4\\ 3 & 1 & 4 & 2 \end{array}\right)$,\\  $Inv = \{(1, 2), (1, 4), (3, 4)\}, sgn(\sigma) = (-1)^{N(\sigma)} = (-1)^3 = -1$\\
\\
\noindent
$\sigma =\left(\begin{array}{cccc} 1 & 2 & 3 & 4\\ 3 & 4 & 1 & 2 \end{array}\right)$,\\  $Inv = \{(1, 3), (1, 4), (2, 3), (2, 4)\}, sgn(\sigma) = (-1)^{N(\sigma)} = (-1)^4 = 1$\\
\\
\noindent
$\sigma =\left(\begin{array}{cccc} 1 & 2 & 3 & 4\\ 3 & 4 & 2 & 1 \end{array}\right)$,\\  $Inv = \{(1, 3), (1, 4), (2, 3), (2, 4), (3, 4)\}, sgn(\sigma) = (-1)^{N(\sigma)} = (-1)^5 = -1$\\
\\
\noindent
$\sigma =\left(\begin{array}{cccc} 1 & 2 & 3 & 4\\ 4 & 2 & 3 & 1 \end{array}\right)$,\\  $Inv = \{(1, 2), (1, 3), (1, 4), (2, 4), (3, 4)\}, sgn(\sigma) = (-1)^{N(\sigma)} = (-1)^5 = -1$\\
\\
\noindent
$\sigma =\left(\begin{array}{cccc} 1 & 2 & 3 & 4\\ 4 & 2 & 1 & 3 \end{array}\right)$,\\  $Inv = \{(1, 2), (1, 3), (1, 4), (2, 3)\}, sgn(\sigma) = (-1)^{N(\sigma)} = (-1)^4 = 1$\\
\\
\noindent
$\sigma =\left(\begin{array}{cccc} 1 & 2 & 3 & 4\\ 4 & 3 & 2 & 1 \end{array}\right)$,\\  $Inv = \{(1, 2), (1, 3), (1, 4), (2, 3), (2, 4), (3, 4)\}, sgn(\sigma) = (-1)^{N(\sigma)} = (-1)^6 = 1$\\
\\
\noindent
$\sigma =\left(\begin{array}{cccc} 1 & 2 & 3 & 4\\ 4 & 3 & 1 & 2 \end{array}\right)$,\\  $Inv = \{(1, 2), (1, 3), (1, 4), (2, 3), (2, 4)\}, sgn(\sigma) = (-1)^{N(\sigma)} = (-1)^5 = -1$\\
\\
\noindent
$\sigma =\left(\begin{array}{cccc} 1 & 2 & 3 & 4\\ 4 & 1 & 3 & 2 \end{array}\right)$,\\  $Inv = \{(1, 2), (1, 3), (1, 4), (3, 4)\}, sgn(\sigma) = (-1)^{N(\sigma)} = (-1)^4 = 1$\\
\\
\noindent
$\sigma =\left(\begin{array}{cccc} 1 & 2 & 3 & 4\\ 4 & 1 & 2 & 3 \end{array}\right)$,\\  $Inv = \{(1, 2), (1, 3), (1, 4)\}, sgn(\sigma) = (-1)^{N(\sigma)} = (-1)^3 = -1$\\

\subsection{Aplicação da fórmula de Leibniz}
A partir do momento que temos tanto as permutações quanto seus sinais, podemos agora aplicar a formula das determinantes.
Relembrando:
\begin{itemize}

\item $A$ é uma matriz quadrada $n$ por $n$ com suas entradas $a_{i,j}$ 
\item $S_n$ é o conjunto de todas as permutações dos números $1, 2, ..., n.$ \item$sgn(\sigma)$ é o sinal da permutação sigma.
\item A somatória é tomada sobre todas as permutações em $S_n$,  e pra cada permutação sigma, tomamos o produto das entradas $a_{i,\sigma(i)}$ de acordo com a diagonal de $A$ com o sinal  $sgn(\sigma)$.
\end{itemize}

\vspace{20pt}
$$
\det(A) = \sum_{\sigma \in S_n} \mathrm{sgn}(\sigma) \prod_{i=1}^n a_{i,\sigma(i)}
$$
\\
\vspace{20pt}
\par Utilizaremos as permutações e abriremos o somatório.
\begin{equation*}
\begin{aligned}
\det(A) = & \sum_{\sigma \in S_4} \mathrm{sgn}(\sigma) \prod_{i=1}^4 a_{i,\sigma(i)} = \\
& + \mathrm{sgn}('1234') \prod_{i=1}^4 a_{i,'1234'(i)}
 + \mathrm{sgn}('1243') \prod_{i=1}^4 a_{i,'1243'(i)} \\
& + \mathrm{sgn}('1324') \prod_{i=1}^4 a_{i,'1324'(i)}
+ \mathrm{sgn}('1342') \prod_{i=1}^4 a_{i,'1342'(i)} \\
& + \mathrm{sgn}('1432') \prod_{i=1}^4 a_{i,'1432'(i)}
+ \mathrm{sgn}('1423') \prod_{i=1}^4 a_{i,'1423'(i)}\\
& + \mathrm{sgn}('2134') \prod_{i=1}^4 a_{i,'2134'(i)}
+ \mathrm{sgn}('2143') \prod_{i=1}^4 a_{i,'2143'(i)}\\
& + \mathrm{sgn}('2314') \prod_{i=1}^4 a_{i,'2314'(i)}
+ \mathrm{sgn}('2341') \prod_{i=1}^4 a_{i,'2341'(i)}\\
& + \mathrm{sgn}('2431') \prod_{i=1}^4 a_{i,'2431'(i)}
+ \mathrm{sgn}('2413') \prod_{i=1}^4 a_{i,'2413'(i)}\\
& + \mathrm{sgn}('3214') \prod_{i=1}^4 a_{i,'3214'(i)}
+ \mathrm{sgn}('3241') \prod_{i=1}^4 a_{i,'3241'(i)}\\
& + \mathrm{sgn}('3124') \prod_{i=1}^4 a_{i,'3124'(i)}
+ \mathrm{sgn}('3142') \prod_{i=1}^4 a_{i,'3142'(i)}\\
& + \mathrm{sgn}('3412') \prod_{i=1}^4 a_{i,'3412'(i)} 
+ \mathrm{sgn}('3421') \prod_{i=1}^4 a_{i,'3421'(i)}\\
& + \mathrm{sgn}('4231') \prod_{i=1}^4 a_{i,'4231'(i)}
+ \mathrm{sgn}('4213') \prod_{i=1}^4 a_{i,'4213'(i)}\\
& + \mathrm{sgn}('4321') \prod_{i=1}^4 a_{i,'4321'(i)}
+ \mathrm{sgn}('4312') \prod_{i=1}^4 a_{i,'4312'(i)}\\
& + \mathrm{sgn}('4132') \prod_{i=1}^4 a_{i,'4132'(i)}
+ \mathrm{sgn}('4123') \prod_{i=1}^4 a_{i,'4123'(i)}\\
\end{aligned}
\end{equation*}

\par Após isso, obteremos a fórmula para as determinantes aplicando o produtório e o sinal correspondente. Com isso a determinante de uma matriz 4 por 4 pode ser determinada por:

\begin{equation*}
\begin{aligned}
det(A) = & + a_{ 1 , 1 }a_{ 2 , 2 }a_{ 3 , 3 }a_{ 4 , 4 } 
- a_{ 1 , 1 }a_{ 2 , 2 }a_{ 3 , 4 }a_{ 4 , 3 } 
- a_{ 1 , 1 }a_{ 2 , 3 }a_{ 3 , 2 }a_{ 4 , 4 } 
+ a_{ 1 , 1 }a_{ 2 , 3 }a_{ 3 , 4 }a_{ 4 , 2 } \\
& - a_{ 1 , 1 }a_{ 2 , 4 }a_{ 3 , 3 }a_{ 4 , 2 } 
+ a_{ 1 , 1 }a_{ 2 , 4 }a_{ 3 , 2 }a_{ 4 , 3 } 
- a_{ 1 , 2 }a_{ 2 , 1 }a_{ 3 , 3 }a_{ 4 , 4 } 
+ a_{ 1 , 2 }a_{ 2 , 1 }a_{ 3 , 4 }a_{ 4 , 3 } \\
& + a_{ 1 , 2 }a_{ 2 , 3 }a_{ 3 , 1 }a_{ 4 , 4 } 
- a_{ 1 , 2 }a_{ 2 , 3 }a_{ 3 , 4 }a_{ 4 , 1 } 
+ a_{ 1 , 2 }a_{ 2 , 4 }a_{ 3 , 3 }a_{ 4 , 1 } 
- a_{ 1 , 2 }a_{ 2 , 4 }a_{ 3 , 1 }a_{ 4 , 3 } \\
& - a_{ 1 , 3 }a_{ 2 , 2 }a_{ 3 , 1 }a_{ 4 , 4 } 
+ a_{ 1 , 3 }a_{ 2 , 2 }a_{ 3 , 4 }a_{ 4 , 1 } 
+ a_{ 1 , 3 }a_{ 2 , 1 }a_{ 3 , 2 }a_{ 4 , 4 } 
- a_{ 1 , 3 }a_{ 2 , 1 }a_{ 3 , 4 }a_{ 4 , 2 } \\
& + a_{ 1 , 3 }a_{ 2 , 4 }a_{ 3 , 1 }a_{ 4 , 2 } 
- a_{ 1 , 3 }a_{ 2 , 4 }a_{ 3 , 2 }a_{ 4 , 1 } 
- a_{ 1 , 4 }a_{ 2 , 2 }a_{ 3 , 3 }a_{ 4 , 1 } 
+ a_{ 1 , 4 }a_{ 2 , 2 }a_{ 3 , 1 }a_{ 4 , 3 } \\
& + a_{ 1 , 4 }a_{ 2 , 3 }a_{ 3 , 2 }a_{ 4 , 1 } 
- a_{ 1 , 4 }a_{ 2 , 3 }a_{ 3 , 1 }a_{ 4 , 2 } 
+ a_{ 1 , 4 }a_{ 2 , 1 }a_{ 3 , 3 }a_{ 4 , 2 } 
- a_{ 1 , 4 }a_{ 2 , 1 }a_{ 3 , 2 }a_{ 4 , 3 }
\end{aligned}
\end{equation*}

\section{Cálculo da determinante para duas matrizes}

Nesta seção, aplicamos a fórmula adquirida na seção anterior para calcular a determinante de duas matrizes diferentes: uma matriz 4x4 com determinante igual a zero e outra matriz 4x4 com determinante igual a um. Após realizar os cálculos necessários, pudemos interpretar os resultados obtidos e compreender melhor a relação entre as características da matriz e o valor de sua determinante. Essa aplicação prática foi muito útil para fixar os conceitos estudados e compreender a relevância da determinante em diversas áreas do conhecimento.
\subsection{Matriz 4 por 4 de determinante 0}
Uma matriz que tem por determinante 0 seria a matriz composta apenas pelo número -1 em seus elementos.

\begin{equation*}
A = 
\begin{bmatrix}
-1 & -1 & -1 & -1\\
-1 & -1 & -1 & -1\\
-1 & -1 & -1 & -1\\
-1 & -1 & -1 & -1\\
\end{bmatrix}
\end{equation*}

Utilizando a fórmula obtida na seção anterior temos:
\newline
\begin{equation*}
\begin{aligned}
det(A) = & + (-1)(-1)(-1)(-1) 
- (-1)(-1)(-1)(-1) \\
& - (-1)(-1)(-1)(-1) 
+ (-1)(-1)(-1)(-1) \\
& - (-1)(-1)(-1)(-1) 
+ (-1)(-1)(-1)(-1) \\
& - (-1)(-1)(-1)(-1)
+ (-1)(-1)(-1)(-1) \\
& + (-1)(-1)(-1)(-1) 
- (-1)(-1)(-1)(-1) \\
& + (-1)(-1)(-1)(-1)
- (-1)(-1)(-1)(-1) \\
& - (-1)(-1)(-1)(-1) 
+ (-1)(-1)(-1)(-1) \\
& + (-1)(-1)(-1)(-1) 
- (-1)(-1)(-1)(-1) \\
& + (-1)(-1)(-1)(-1) 
- (-1)(-1)(-1)(-1) \\
& - (-1)(-1)(-1)(-1) 
+ (-1)(-1)(-1)(-1) \\
& + (-1)(-1)(-1)(-1) 
- (-1)(-1)(-1)(-1) \\
& + (-1)(-1)(-1)(-1)
- (-1)(-1)(-1)(-1)
\end{aligned}
\end{equation*}

\begin{equation*}
\begin{aligned}
det(A) = & + 1 - 1 - 1 + 1 - 1 + 1 - 1 + 1 \\
& + 1 - 1  + 1 - 1 - 1 + 1 + 1 - 1 \\ 
& + 1 - 1 - 1 + 1 + 1 - 1 + 1 - 1
\\\\
det(A) = 0
\end{aligned}
\end{equation*}
\subsubsection{Verificação no código Python}
Obtem-se, no código python, como determinante da tabela descrita acima:
\subsection{Matriz 4 por 4 de determinante diferente de zero: determinante 1}
Uma matrizes que tem por determinante 1 seria a matriz composta apenas pelo número -1 em seus elementos $a_{1,4}, a_{2,3},a_{3,2},a_{4,1}$ e 0 nos elementos restantes. Os elementos com -1 formam uma das diagonais com sinal positivo na fórmula de Leibniz.
\begin{equation*}
A = 
\begin{bmatrix}
0 & 0 & 0 & -1\\
0 & 0 & -1 & 0\\
0 & -1 & 0 & 0\\
-1 & 0 & 0 & 0\\
\end{bmatrix}
\end{equation*}
\\
Utilizando a fórmula obtida na seção anterior temos:\\
\newline
\begin{equation*}
\begin{aligned}
det(A) = & + (0)(0)(0)(0) 
- (0)(0)(0)(0)  
- (0)(-1)(-1)(0) 
+ (0)(-1)(0)(0) \\
& - (0)(0)(0)(0)  
+ (0)(0)(-1)(0) 
-(0)(0)(0)(0)  
+ (0)(0)(0)(0)  \\
& + (0)(-1)(0)(0) 
- (0)(-1)(0)(-1) 
+ (0)(0)(0)(-1) 
- (0)(0)(0)(0)  \\
& - (0)(0)(0)(0)  
+ (0)(0)(0)(-1) 
+ (0)(0)(-1)(0)
- (0)(0)(0)(0)  \\
& + (0)(0)(0)(0) 
- (0)(0)(-1)(-1) 
- (-1)(0)(0)(-1) 
+ (-1)(0)(0)(0) \\
& + (-1)(-1)(-1)(-1) 
- (-1)(-1)(0)(0)
+ (-1)(0)(0)(0) 
- (-1)(0)(-1)(0)
\end{aligned}
\end{equation*}
\\\\
\begin{equation*}
det(A) = + (-1)(-1)(-1)(-1) = 1
\end{equation*}
\subsubsection{Verificação no código Python}
Obtem-se, no código python, como determinante da tabela descrita acima:
\section{Conclusão}

O estudo da determinante é fundamental para a compreensão e aplicação de conceitos importantes da álgebra linear. Durante a elaboração desse trabalho, compreendemos melhor a fórmula de Leibniz e sua aplicação para o cálculo da determinante de uma matriz 4x4. Por fim, disponibilizamos o repositório do Github: 
\begin{center}
\url{ https://github.com/feg59crz/determinante_matriz }
\end{center}
onde se encontra o arquivo latex 'main.tex' do presente texto e a implementação da fórmula da determinante de qualquer matriz quadrada em um código python 'main.py' junto com o resultado do console das duas matrizes, para que os interessados possam acessar e verificar os resultados obtidos.
\end{document}
